\documentclass[11.5pt]{article}
\usepackage{graphics, graphicx, cite, fancybox, setspace}
\usepackage{amsfonts, amssymb, amsmath, latexsym, epic, eepic, url}
\usepackage{algorithm}
\usepackage{algorithmic}
\usepackage[letterpaper, left=1in, right=1in, top=1in, bottom=1in]{geometry}

\begin{document}

\title{ECEC-413: Introduction to Parallel Computer Architecture \\
CUDA Programming Lab 2}
\author{Prof. Naga Kandasamy, ECE Department, Drexel University}
\maketitle %
\date{}

\noindent The lab is due November 30, 2012. You can work on this lab in teams of up to two people.
\vspace{12pt}

\noindent \textbf{Vector Dot Product.} Given two $n$-element vectors \textbf{a} and \textbf{b}, their dot product $\textbf{a} \cdot \textbf{b}$ is given by
\begin{equation*}
\textbf{a} \cdot \textbf{b} = \sum_{i = 1}^{n}a_ib_i,
\end{equation*}
\noindent where $a_i$ and $b_i$ denote the $i^{\texttt{th}}$ elements of vectors \textbf{a} and \textbf{b}, respectively. \vspace{6pt}

\noindent The program provided to you accepts no arguments. It creates two randomly initialized vectors and computes their dot product using both the CPU and the GPU. The solution provided by the GPU is compared to that generated by the CPU. If the solutions match within a certain tolerance, the application will print out ``Test PASSED'' to the screen before exiting. \vspace{6pt}

\noindent Please answer the following questions:
\begin{itemize}
\item \textbf{(25 points)} Edit the \texttt{computeOnDevice()} function within the file \texttt{vector\_dot\_product.cu} to complete the functionality of vector dot product on the GPU. You may add multiple kernels to the \texttt{vector\_dot\_product\_kernel.cu} file to achieve this functionality. Do not change the source code elsewhere (except for adding timing-related code). The size of the vectors is guaranteed to be 10 million elements. The CUDA source files for this question are available in a zip file called \texttt{lab\_2.zip}.

\item \textbf{(5 points)} Provide a two/three page report describing: (1) the design of your kernel (use code or pseudocode to clarify the discussion); (2) the speedup achieved over the serial version; and (3) sensitivity of the kernel to thread-block size in terms of the execution time.
\end{itemize}

\noindent Provide a \underline{hard copy} of your report and email the files needed to run your code to me as a single zip file called \texttt{lab\_2.zip}.

\end{document}
