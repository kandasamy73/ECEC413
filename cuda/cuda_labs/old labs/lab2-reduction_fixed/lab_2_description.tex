\documentclass[11.5pt]{article}
\usepackage{graphics, graphicx, cite, fancybox, setspace}
\usepackage{amsfonts, amssymb, amsmath, latexsym, epic, eepic, url}
\usepackage{algorithm}
\usepackage{algorithmic}
\usepackage[letterpaper, left=1in, right=1in, top=1in, bottom=1in]{geometry}
%\usepackage{setspace}
\begin{document}

%% Different font in captions
\newcommand{\captionfonts}{\bf}{\small}
\makeatletter  % Allow the use of @ in command names
\long\def\@makecaption#1#2{%
  \vskip\abovecaptionskip
  \sbox\@tempboxa{{\captionfonts #1: #2}}%
  \ifdim \wd\@tempboxa >\hsize
    {\captionfonts #1: #2\par}
  \else
    \hbox to\hsize{\hfil\box\@tempboxa\hfil}%
  \fi
  \vskip\belowcaptionskip}
\makeatother   % Cancel the effect of \makeatletter
\renewcommand{\figurename}{Fig.} %Make the caption read Fig.

\title{ECEC 413: Introduction to Parallel Computer Architecture \\ CUDA Lab 2: Data Parallel Reduction}
\author{Prof. Naga Kandasamy, ECE Department, Drexel University}
\maketitle %
\date{}

\noindent The Lab is due March 1, 2010. You may work on the assignment in teams of up to two people.
\vspace{12pt}

\noindent \textbf{Important:} Before starting the lab assignment, please read the tutorial on getting the CUDA SDK working on the cluster. You will find the tutorial (written by Tony Shackleford) on webCT under the ``Labs'' folder. \vspace{12pt}

\noindent Your lab assignment comprises of \underline{two parts} as described below:
\begin{itemize}
\item \textbf{Part 1:} Edit the source files \texttt{vector\_reduction.cu} and \texttt{vector\_reduction\_kernel.cu} to complete the functionality of the parallel addition reduction on the device using the technique shown in Fig.~6.2 in the textbook.  The size of the array is guaranteed to be equal to 512 elements for this assignment. The array must be loaded from GPU global memory. The use of shared memory is not required for this assignment.
    
\item \textbf{Part 2:} Repeat Part 1, except in this case, use the kernel design shown in Fig.~6.4.
\end{itemize}   

\noindent Your program should accept no arguments. The application will create a randomly initialized array to
    process. After the GPU kernel is invoked, it will compute the correct solution value using the CPU, and compare that solution with the GPU-computed solution.  If the solutions match (within a certain tolerance), it will print out ``Test PASSED'' to the screen before exiting. \vspace{12pt}
    
\noindent You must e-mail me \underline{all} of the files needed to compile and run your code as separate zip files---one zip file for Part 1 and one zip file for Part 2.
\vspace{12pt}

\noindent Also, once you have completed the assignment, answer the following questions in a separate lab report.
\begin{itemize}
\item Compare the execution times achieved by the two different reduction kernels and comment on why you think one kernel performs better than the other.

\item Answer this question for Parts 1 and 2: How many times does your thread block synchronize to reduce the array of 512 elements to a single value?

\item Answer this question for Parts 1 and 2: What is the minimum, maximum, and average number of useful operations that a thread will perform?  Useful operations are those that directly contribute to the final reduction value.
\end{itemize}
\vspace{12pt}

\noindent Your assignment will be graded on the following parameters:
\begin{itemize}
\item Correctness: 25\%.
\item Functionality: 30\%. Your kernels uses an $O(n)$ operation data-parallel reduction algorithm.
\item Report: 45\%. Answer to question 1: 25\%, answer to question 2: 10\%, answer to question 3: 10\%
\end{itemize}
\end{document}
