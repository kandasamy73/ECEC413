\documentclass[12pt]{article}
\usepackage{graphics, graphicx, cite, fancybox, setspace}
\usepackage{amsfonts, amssymb, amsmath, latexsym, epic, eepic, url}
\usepackage{algorithm}
\usepackage{algorithmic}
\usepackage[letterpaper, left=1in, right=1in, top=1in, bottom=1in]{geometry}
\usepackage{times}
\usepackage[parfill]{parskip}

\begin{document}

\title{ECEC 413/622: Parallel Computer Architecture \\
Vector Dot Product}
\author{Instructor: Prof. Naga Kandasamy \\ 
    ECE Department, Drexel University}
\maketitle %
\date{}

\noindent The lab is due March 19, 2017, via BBLearn. You can work on this lab in teams of up to two people.
\vspace{12pt}

\textbf{(10 points)} Given two $n$-element vectors \textbf{a} and \textbf{b}, their dot product $\textbf{a} \cdot \textbf{b}$ is given by
\begin{equation*}
\textbf{a} \cdot \textbf{b} = \sum_{i = 1}^{n}a_ib_i,
\end{equation*}
where $a_i$ and $b_i$ denote the $i^{\texttt{th}}$ elements of vectors \textbf{a} and \textbf{b}, respectively. 

The program provided to you accepts $n$ as an argument. It then creates two randomly initialized vectors and computes their dot product using both the CPU and the GPU. The solution provided by the GPU will be compared to that generated by the CPU. If the solutions match within the specified tolerance, it will print out ``Test PASSED'' to the screen before exiting.

Edit the \texttt{computeOnDevice()} function within the file \texttt{vector\_dot\_product.cu} to complete the functionality of vector dot product on the GPU. You may develop multiple kernels to achieve this functionality. The program will be graded based on how efficiently it uses shared memory and/or texture memory, and atomic operations on the GPU.

Provide a brief report describing the design of your kernel, using code or pseudocode to clarify the discussion, as well as the speedup achieved over the serial version for vector sizes of $10^5$, $10^6$, and $10^7$ elements. Also quantify the sensitivity of the kernel to thread-block size in terms of the execution time.

Upload your report as well as the files needed to run your code as a single zip file on BBLearn.

\end{document}
